% задание и сама лабораторная работа
\newpage
\section*{Задание}
\addcontentsline{toc}{section}{\tocsecindent{Задание}}
\Large{Лабораторная работа №5}

\begin{itemize}
	\item Если запустить интерпретатор и ввести:
		\begin{itemize}
			\item (setf  a 2)
			\item (setf b 3)
			\item (setf c 5)
		\end{itemize}
	\item Что будет на экране, если вводить:
		\begin{itemize}
			\item a ; 2
			\item b ; 3
			\item c ; 5
			\item ‘a ; a
			\item ‘(+ a c) ; (+ a c)
			\item (a) ; Eval error
			\item (eval ‘a) ; 2
		\end{itemize}
	\underline{Добавив:}\\
	\textbf{(defun a() ‘b) }\\
	\textbf{(defun b() 4)}
	\item Что будет на экране, если вводить:
		\begin{itemize}
			\item a ; 2
			\item b ; 3
			\item c ; 5
			\item ‘a ; a
			\item ‘(+ a c) ; (+ a c)
			\item (a) ; b
			\item (eval ‘a) ; 2
			\item (a) ; 4
			\item (+ a a) ; 4
			\item (+ (b) b) ; 7
			\item (b b b) ; Eval error
		\end{itemize}
	\underline{Добавив:}\\
	\textbf{(setf a b)}
	\item Что будет на экране, если вводить:
		\begin{itemize}
			\item a ; 3
			\item b ; 3
			\item c ; 5
			\item `a ; a
			\item `(+ a c) ; (+ a c)
			\item (a) ; b
			\item (eval `a) ; 3
		\end{itemize}
	\underline{Добавив:}\\
	\textbf{(setf c b)}
	\item Что будет на экране, если вводить:
		\begin{itemize}
			\item a ; 3
			\item b ; 3
			\item c ; 3
			\item `a ; a
			\item `(+ a c) ; (+ a c)
			\item (a) ; b
			\item (eval `a) ; 3
		\end{itemize}
	\underline{Добавив:}\\
	\textbf{(defun a(x y) (+ x y))}
	\item Что будет на экране, если вводить:
		\begin{itemize}
			\item a ; 3
			\item b ; 3
			\item c ; 3
			\item `a ; a
			\item `(+ a c) ; (+ a c)
			\item (a) ; Eval error
			\item (eval `a) ; 3
			\item (a a a) ; 6
			\item (a b a) ; 6
		\end{itemize}
	\hrulefill 
\begin{itemize}
	\item (defun b(x y) (setf b (+ b 1)) (* x y b) ) 
\end{itemize}
	\hrulefill
	%	\hrulefill
	\item Что будет на экране, если вводить:
		\begin{itemize}
			\item a ; 3
			\item b ; 3
			\item (b b b) ; 36
		\end{itemize}
	\item Напишите функцию, которая вычисляет катет по гипотенузе и другому катету.
	\lstinputlisting[
		language=Lisp,                 % выбор языка для подсветки (здесь это С)
		basicstyle=\small\sffamily, % размер и начертание шрифта для подсветки кода
		numbers=left,               % где поставить нумерацию строк (слева\справа)
		numberstyle=\tiny,           % размер шрифта для номеров строк
		stepnumber=1,                   % размер шага между двумя номерами строк
		numbersep=5pt,                % как далеко отстоят номера строк от подсвечиваемого кода
		backgroundcolor=\color{white}, % цвет фона подсветки - используем \usepackage{color}
		showspaces=false,            % показывать или нет пробелы специальными отступами
		showstringspaces=false,      % показывать или нет пробелы в строках
		showtabs=false,             % показывать или нет табуляцию в строках
		frame=single,              % рисовать рамку вокруг кода
		tabsize=2,                 % размер табуляции по умолчанию равен 2 пробелам
		captionpos=t,              % позиция заголовка вверху [t] или внизу [b] 
		breaklines=true,           % автоматически переносить строки (да\нет)
		breakatwhitespace=false, % переносить строки только если есть пробел
		escapeinside={\%*}{*)},  % если нужно добавить комментарии в коде
		caption=Функция для вычисление катета по заданной гипотенузе и другому катету в прямоугольном треугольнике.
		]{./leg}
		
\end{itemize}