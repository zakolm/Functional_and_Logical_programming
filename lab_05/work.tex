% задание и сама лабораторная работа
\newpage
\section*{Задание}
\addcontentsline{toc}{section}{\tocsecindent{Задание}}
\Large{Лабораторная работа №5}

\begin{itemize}
	\item Если запустить интерпретатор и ввести:
		\begin{itemize}
			\item (setf  a 2)
			\item (setf b 3)
			\item (setf c 5)
		\end{itemize}
	\item Что будет на экране, если вводить:
		\begin{itemize}
			\item a
			\item b
			\item c
			\item ‘a
			\item ‘(+ a c)
			\item (a)
			\item (eval ‘a)
		\end{itemize}
	\underline{Добавив:}\\
	\textbf{(defun a() ‘b) }\\
	\textbf{(defun b() 4)}
	\item Что будет на экране, если вводить:
		\begin{itemize}
			\item a
			\item b 
			\item c
			\item ‘a
			\item ‘(+ a c)
			\item (a)
			\item (eval ‘a)
			\item (a)
			\item (+ a a)
			\item (+ (b) b)
			\item (b b b)
		\end{itemize}
	\underline{Добавив:}\\
	\textbf{(setf a b)}
	\item Что будет на экране, если вводить:
		\begin{itemize}
			\item a
			\item b 
			\item c
			\item `a
			\item `(+ a c)
			\item (a)
			\item (eval `a)
		\end{itemize}
	\underline{Добавив:}\\
	\textbf{(setf c b)}
	\item Что будет на экране, если вводить:
		\begin{itemize}
			\item a
			\item b
			\item c
			\item `a
			\item `(+ a c)
			\item (a)
			\item (eval `a)
		\end{itemize}
	\underline{Добавив:}\\
	\textbf{(defun a(x y) (+ x y))}
	\item Что будет на экране, если вводить:
		\begin{itemize}
			\item a
			\item b
			\item c
			\item `a
			\item `(+ a c)
			\item (a)
			\item (eval `a)
			\item (a a a)
			\item (a b a)
		\end{itemize}
	\hrulefill 
\begin{itemize}
	\item (defun b(x y) (setf b (+ b 1)) (* x y b) ) 
\end{itemize}
	\hrulefill
	%	\hrulefill
	\item Что будет на экране, если вводить:
		\begin{itemize}
			\item a
			\item b
			\item (b b b)
		\end{itemize}
	\item Напишите функцию, которая вычисляет катет по гипотенузе и другому катету.
\end{itemize}