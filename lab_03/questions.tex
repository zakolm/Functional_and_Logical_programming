% ответы на вопросы
\newpage
\section*{Ответы на вопросы}
\addcontentsline{toc}{section}{\tocsecindent{Ответы на вопросы}}
\begin{enumerate}
    \item\Large{Как представляются списки в ОП? Выписать и дать определение основным элементам языка Lisp.}\\
         Списочная ячейка состоит из двух частей, полей first и rest. Каждое из полей содержит указатель. Указатель может ссылаться на другую списочную ячейку или на некоторый другой Lisp объект, как, например, атом. Указатели между ячейками образуют как бы цепочку, по которой можно из предыдущей ячейки попасть в следующую и так, наконец, до атомарных объектов. Каждый известный системе атом записан в определённом месте памяти лишь один раз.\\
    \underline{Основные элементы языка:}
        \begin{itemize}
            \item Атомы:
                \begin{itemize}
                    \item Символ
                    \item Символьная константа
                    \item Число(целые, действительные, рациональные)
                    \item Логические(T, Nil)
                \end{itemize}
            \item Точечные пары:
                \begin{itemize}
                    \item (<атом> . <атом>)
                    \item (<точ. пара> . <атом>)
                    \item (<атом> . <точ. пара>)
                    \item (<точ. пара> . <точ. пара>)
                \end{itemize}
            \item S-выражения:
                \begin{itemize}
                    \item <атом> | <точ.пара>
                \end{itemize}
            \item Списки:
                \begin{itemize}
                    \item (<s-выражение> . <список>)
                    \item (<пустой список>) $\equiv$ Nil
                \end{itemize}
        \end{itemize}
    \item\Large{Как выполняются CAR и CDR?}
        \begin{itemize}
            \item car - переходит по car указателю и возвращает голову
            \item cdr - переходит по cdr указателю и возвращает хвост(остаток)
        \end{itemize}
    \item Классификация функций:
        \begin{enumerate}
            \item Чистые(математические)- функции имеют фиксированное количество аргументов;
            \item Специальные- специальным образом обрабатывают свои аргументы;
            \item Псевдофункции- реализация аппаратно-зивисимых действий;
            \item Функции, допускающие вариантные значения- в процессе работы могут быть выбран один из результатов;
            \item Функции, позволяющие организовать отложенные "ленивые" вычисления- ленивые вычисления могут быть вообще не выполнены;
            \item Функции высших порядков- в процессе работы формируют другие функции.
        \end{enumerate}
    \item Классификация базисных функций:
        \begin{enumerate}
            \item Конструкторы(cons, list)
                \begin{itemize}
                    \item \textbf{cons}- двухъаргументная функция, создаёт одну списковую ячейку и расставляет указатели;
                        \\ \textbf{(cons 'a 'b) -> (a.b)} 
                        \\ \textbf{(cons 'a '(b)) -> (a b)}
                    \item \textbf{list}- создаёт столько списковых ячеек, сколько аргументов
                        \\ \textbf{(list 'a 'b) -> (a b)}
                    \\
                    cons работает быстрее, чем list, но может быть организован не список
                \end{itemize}
            \item Селекторы(car, cdr)
                \begin{itemize}
                    \item \textbf{car}- переходит по car указателю и возвращает голову
                    \item \textbf{cdr}- переходит по cdr указателю и возвращает хвост(остаток)
                \end{itemize}
            \item Предикаты(позволяющие определить структуру элементов)
                \begin{itemize}
                    \item \textbf{atom}- возвращает T, если аргумент- атом, иначе Nil;
                    \item \textbf{listp}-  возвращает T, если аргумент- список, иначе Nil;
                    \item \textbf{consp}- представлена ли пара в виде списковых ячеек;
                    \item \textbf{numberp}- возвращает T, если аргумент- число, иначе Nil;
                    \item \textbf{symbolp}- возвращает T, если аргумент- не число, иначе Nil;
                \end{itemize}
            \item Сравнение(позволяющие сравнивать элементы):
                \begin{itemize}
                    \item \textbf{базовая eq}- сравнивает два символьных атома(указатели на них) на их тождественность возвращает Nil или T, применима только для символов.
                        \\ \textbf{(eq T T) -> T}
                    \item \textbf{eql}- сравнивает два числа(по форме их представления)
                        \\ \textbf{(eql 3 3) -> T}
                        \\ \textbf{(eql 3 3.0) -> Nil}
                    \item \textbf{=}- сравнивает два числа
                        \\ \textbf{(= 3 3) -> T}
                        \\ \textbf{(= 3 3.0) -> T}
                    \item \textbf{equal}- сравнивает как и eql, но также может сравнивать списки
                    \item \textbf{equalp}- сравнивает все варианты представления, но долго работает
                \end{itemize}
            \underline{Для чисел:}
                \begin{itemize}
                    \item \textbf{oddp}- проверка на нечетность
                    \item \textbf{evenp}- проверка на четность
                \end{itemize}
        \end{enumerate}
    \item В чем отличие выполнения функций list и cons?
    	 \begin{itemize}
        		\item \textbf{cons}- является базисом языка, она на вход принимает ровно два аргумента и создает одну списковую ячейку(расставляет указатели)
        		\item \textbf{list}- написана на базе функции \textbf{cons}, принимает любое количество аргументов и создает список
	\end{itemize}
\end{enumerate}

% ответ
%\\