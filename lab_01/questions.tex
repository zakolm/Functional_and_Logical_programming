% ответы на вопросы
\newpage
\section*{Ответы на вопросы}
\addcontentsline{toc}{section}{\tocsecindent{Ответы на вопросы}}
\begin{enumerate}
  \item\Large{Как воспринимается символ '? Дать определение списка.}
    \begin{itemize}
    Функция quote блокирует вычисления(предохраняет свой единственный аргумент от вычисления):
    \begin{itemize}
    	\item (quote (a b c)) блокирует вычисление a b c\\
	(a b c)  система должна вычислить a b c
	\item '(a b c) аналог quote(блокирование запуска eval)\\
	a b c тогда воспринимаются как данные, а не как фрагмент программы
    \end{itemize}
      \item \underline{Список}- рекурсивно определенная динамическая структура, которая может быть пустой.
      \item \underline{Список}- динамическая структура, которая может быть пустой, имеет голову и хвост
    \end{itemize}
  \item\Large{Как представляются списки в ОП? Выписать и дать определение основным элементам языка Lisp.}\\
  Списочная ячейка состоит из двух частей, полей first и rest. Каждое из полей содержит указатель. Указатель может ссылаться на другую списочную ячейку или на некоторый другой Lisp объект, как, например, атом. Указатели между ячейками образуют как бы цепочку, по которой можно из предыдущей ячейки попасть в следующую и так, наконец, до атомарных объектов. Каждый известный системе атом записан в определённом месте памяти лишь один раз. (
  \\
    \underline{Основные элементы языка:}
    \begin{itemize}
      \item Атомы:
      \begin{itemize}
          \item Символ
          \item Символьная константа
	  \item Число(целые, действительные, рациональные)
	  \item Логические(T, Nil)
      \end{itemize}
      \item Точечные пары:
      \begin{itemize}
      	   \item (<атом> . <атом>)
	   \item (<точ. пара> . <атом>)
	   \item (<атом> . <точ. пара>)
	   \item (<точ. пара> . <точ. пара>)
      \end{itemize}
      \item S-выражения:
      \begin{itemize}
      	   \item <атом> | <точ.пара>
      \end{itemize}
      \item Списки:
      \begin{itemize}
      	   \item (<s-выражение> . <список>)
	   \item (<пустой список>) $\equiv$ Nil
      \end{itemize}
    \end{itemize}
\end{enumerate}

% ответ
%\Large{3) Как выполняются CAR и CDR и какой результат они вернут в разных случаях?}
%\\