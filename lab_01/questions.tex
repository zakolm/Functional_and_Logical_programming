% ответы на вопросы
\newpage
\section*{Ответы на вопросы}
\addcontentsline{toc}{section}{\tocsecindent{Ответы на вопросы}}
\begin{enumerate}
  \item\Large{Как воспринимается символ '? Дать определение списка.}
    \begin{itemize}
      \item \underline{Список}- рекурсивно определенная структура, которая может быть пустой.
      \item \underline{Список}- 
    \end{itemize}
  \item\Large{Как представляются списки в ОП? Выписать и дать определение основным элементам языка Lisp.}\\
  \\
    \underline{Основные элементы языка:}
    \begin{itemize}
      \item Атомы:
      \begin{itemize}
          \item Символ
          \item Символьная константа
	  \item Число(целые, действительные, рациональные)
	  \item Логические(T, Nil)
      \end{itemize}
      \item Точечные пары:
      \begin{itemize}
      	   \item (<атом> . <атом>)
	   \item (<точ. пара> . <атом>)
	   \item (<атом> . <точ. пара>)
	   \item (<точ. пара> . <точ. пара>)
      \end{itemize}
      \item S-выражения:
      \begin{itemize}
      	   \item <атом> | <точ.пара>
      \end{itemize}
      \item Списки:
      \begin{itemize}
      	   \item (<s-выражение> . <список>)
	   \item (<пустой список>) $\equiv$ Nil
      \end{itemize}
    \end{itemize}
\end{enumerate}

% ответ
%\Large{3) Как выполняются CAR и CDR и какой результат они вернут в разных случаях?}
%\\