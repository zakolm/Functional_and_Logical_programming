% задание и сама лабораторная работа
\newpage
\section*{Задание}
\addcontentsline{toc}{section}{\tocsecindent{Задание}}
\Large{Лабораторная работа №6}

\begin{itemize}
	\item Написать функцию, которая переводит температуру в системе Фаренгейта в температуру по Цельсию (defun f-to-c (temp) ...).\\
	\underline{Формулы: c = 5/9*(f-32.0); f = 9/5*c+32.0.}\\
	Как бы назывался роман Р.Бредбери "451 градус по Фарингейту" в системе по Цельсию?
	\lstinputlisting[
		language=Lisp,                 % выбор языка для подсветки (здесь это С)
		basicstyle=\small\sffamily, % размер и начертание шрифта для подсветки кода
		numbers=left,               % где поставить нумерацию строк (слева\справа)
		numberstyle=\tiny,           % размер шрифта для номеров строк
		stepnumber=1,                   % размер шага между двумя номерами строк
		numbersep=5pt,                % как далеко отстоят номера строк от подсвечиваемого кода
		backgroundcolor=\color{white}, % цвет фона подсветки - используем \usepackage{color}
		showspaces=false,            % показывать или нет пробелы специальными отступами
		showstringspaces=false,      % показывать или нет пробелы в строках
		showtabs=false,             % показывать или нет табуляцию в строках
		frame=single,              % рисовать рамку вокруг кода
		tabsize=2,                 % размер табуляции по умолчанию равен 2 пробелам
		captionpos=t,              % позиция заголовка вверху [t] или внизу [b] 
		breaklines=true,           % автоматически переносить строки (да\нет)
		breakatwhitespace=false, % переносить строки только если есть пробел
		escapeinside={\%*}{*)},  % если нужно добавить комментарии в коде
		caption=Функция для вычисление катета по заданной гипотенузе и другому катету в прямоугольном треугольнике.
		]{./f-to-c}
	Роман назывался бы "232.7778 градусов по Цельсию" ; (f-to-c 451)
	\item Что получится при вычислении каждого из выражений?
		\begin{enumerate}
			\item (list 'cons t Nil) = (cons t Nil)
			\item (eval (eval (list 'cons t Nil))) = *** Eval error ***
			\item (apply \#'cons '(t Nil)) = (t.Nil) = (t)
			\item (list 'eval Nil) = (eval Nil)
			\item (eval (list 'cons t Nil)) = (t.Nil) = (t)
			\item (eval Nil) = Nil
			\item	(eval (list 'eval Nil)) = Nil
		\end{enumerate}
	\item Написать функцию, которая принимает целое число и возвращает первое четное число, не меньше аргумента.
	\lstinputlisting[
		language=Lisp,                 % выбор языка для подсветки (здесь это С)
		basicstyle=\small\sffamily, % размер и начертание шрифта для подсветки кода
		numbers=left,               % где поставить нумерацию строк (слева\справа)
		numberstyle=\tiny,           % размер шрифта для номеров строк
		stepnumber=1,                   % размер шага между двумя номерами строк
		numbersep=5pt,                % как далеко отстоят номера строк от подсвечиваемого кода
		backgroundcolor=\color{white}, % цвет фона подсветки - используем \usepackage{color}
		showspaces=false,            % показывать или нет пробелы специальными отступами
		showstringspaces=false,      % показывать или нет пробелы в строках
		showtabs=false,             % показывать или нет табуляцию в строках
		frame=single,              % рисовать рамку вокруг кода
		tabsize=2,                 % размер табуляции по умолчанию равен 2 пробелам
		captionpos=t,              % позиция заголовка вверху [t] или внизу [b] 
		breaklines=true,           % автоматически переносить строки (да\нет)
		breakatwhitespace=false, % переносить строки только если есть пробел
		escapeinside={\%*}{*)},  % если нужно добавить комментарии в коде
		caption=Функция для вычисление катета по заданной гипотенузе и другому катету в прямоугольном треугольнике.
		]{./next-even}
	\hfil{}\\
	\item Написать функцию, которая принимает число и возвращает число того же знака на 1 больше модуля аргумента.
	\lstinputlisting[
		language=Lisp,                 % выбор языка для подсветки (здесь это С)
		basicstyle=\small\sffamily, % размер и начертание шрифта для подсветки кода
		numbers=left,               % где поставить нумерацию строк (слева\справа)
		numberstyle=\tiny,           % размер шрифта для номеров строк
		stepnumber=1,                   % размер шага между двумя номерами строк
		numbersep=5pt,                % как далеко отстоят номера строк от подсвечиваемого кода
		backgroundcolor=\color{white}, % цвет фона подсветки - используем \usepackage{color}
		showspaces=false,            % показывать или нет пробелы специальными отступами
		showstringspaces=false,      % показывать или нет пробелы в строках
		showtabs=false,             % показывать или нет табуляцию в строках
		frame=single,              % рисовать рамку вокруг кода
		tabsize=2,                 % размер табуляции по умолчанию равен 2 пробелам
		captionpos=t,              % позиция заголовка вверху [t] или внизу [b] 
		breaklines=true,           % автоматически переносить строки (да\нет)
		breakatwhitespace=false, % переносить строки только если есть пробел
		escapeinside={\%*}{*)},  % если нужно добавить комментарии в коде
		caption=Функция для вычисление катета по заданной гипотенузе и другому катету в прямоугольном треугольнике.
		]{./next-abs-number}
	\item Написать функцию, которая принимает два числа и возвращает список из этих чисел, расположенный по возрастанию.
	\lstinputlisting[
		language=Lisp,                 % выбор языка для подсветки (здесь это С)
		basicstyle=\small\sffamily, % размер и начертание шрифта для подсветки кода
		numbers=left,               % где поставить нумерацию строк (слева\справа)
		numberstyle=\tiny,           % размер шрифта для номеров строк
		stepnumber=1,                   % размер шага между двумя номерами строк
		numbersep=5pt,                % как далеко отстоят номера строк от подсвечиваемого кода
		backgroundcolor=\color{white}, % цвет фона подсветки - используем \usepackage{color}
		showspaces=false,            % показывать или нет пробелы специальными отступами
		showstringspaces=false,      % показывать или нет пробелы в строках
		showtabs=false,             % показывать или нет табуляцию в строках
		frame=single,              % рисовать рамку вокруг кода
		tabsize=2,                 % размер табуляции по умолчанию равен 2 пробелам
		captionpos=t,              % позиция заголовка вверху [t] или внизу [b] 
		breaklines=true,           % автоматически переносить строки (да\нет)
		breakatwhitespace=false, % переносить строки только если есть пробел
		escapeinside={\%*}{*)},  % если нужно добавить комментарии в коде
		caption=Функция для вычисление катета по заданной гипотенузе и другому катету в прямоугольном треугольнике.
		]{./growth-list}
	\item Написать функцию, которая принимает три числа и возвращает T только тогда, когда первое число расположенно между вторым и третьим.
	\lstinputlisting[
		language=Lisp,                 % выбор языка для подсветки (здесь это С)
		basicstyle=\small\sffamily, % размер и начертание шрифта для подсветки кода
		numbers=left,               % где поставить нумерацию строк (слева\справа)
		numberstyle=\tiny,           % размер шрифта для номеров строк
		stepnumber=1,                   % размер шага между двумя номерами строк
		numbersep=5pt,                % как далеко отстоят номера строк от подсвечиваемого кода
		backgroundcolor=\color{white}, % цвет фона подсветки - используем \usepackage{color}
		showspaces=false,            % показывать или нет пробелы специальными отступами
		showstringspaces=false,      % показывать или нет пробелы в строках
		showtabs=false,             % показывать или нет табуляцию в строках
		frame=single,              % рисовать рамку вокруг кода
		tabsize=2,                 % размер табуляции по умолчанию равен 2 пробелам
		captionpos=t,              % позиция заголовка вверху [t] или внизу [b] 
		breaklines=true,           % автоматически переносить строки (да\нет)
		breakatwhitespace=false, % переносить строки только если есть пробел
		escapeinside={\%*}{*)},  % если нужно добавить комментарии в коде
		caption=Функция для вычисление катета по заданной гипотенузе и другому катету в прямоугольном треугольнике.
		]{./first-middle2}
	\item Каков результат вычисления следующих выражений?
		\begin{enumerate}
			\item (and 'fee 'fie 'foe) = foe
			\item (or 'fee 'fie 'foe) = fee
			\item (and (equal 'abc 'abc) 'yes) = yes
			\item (or Nil 'fie 'foe) = fie
			\item (and Nil 'fie 'foe) = Nil
			\item (or (equal 'abc 'abc) 'yes) = T
		\end{enumerate}
	\item Решить задачу 4, используя для ее решения конструкции IF, COND, AND/OR.
		\lstinputlisting[
		language=Lisp,                 % выбор языка для подсветки (здесь это С)
		basicstyle=\small\sffamily, % размер и начертание шрифта для подсветки кода
		numbers=left,               % где поставить нумерацию строк (слева\справа)
		numberstyle=\tiny,           % размер шрифта для номеров строк
		stepnumber=1,                   % размер шага между двумя номерами строк
		numbersep=5pt,                % как далеко отстоят номера строк от подсвечиваемого кода
		backgroundcolor=\color{white}, % цвет фона подсветки - используем \usepackage{color}
		showspaces=false,            % показывать или нет пробелы специальными отступами
		showstringspaces=false,      % показывать или нет пробелы в строках
		showtabs=false,             % показывать или нет табуляцию в строках
		frame=single,              % рисовать рамку вокруг кода
		tabsize=2,                 % размер табуляции по умолчанию равен 2 пробелам
		captionpos=t,              % позиция заголовка вверху [t] или внизу [b] 
		breaklines=true,           % автоматически переносить строки (да\нет)
		breakatwhitespace=false, % переносить строки только если есть пробел
		escapeinside={\%*}{*)},  % если нужно добавить комментарии в коде
		caption=Функция для вычисление катета по заданной гипотенузе и другому катету в прямоугольном треугольнике.
		]{./first-middle}
		\lstinputlisting[
		language=Lisp,                 % выбор языка для подсветки (здесь это С)
		basicstyle=\small\sffamily, % размер и начертание шрифта для подсветки кода
		numbers=left,               % где поставить нумерацию строк (слева\справа)
		numberstyle=\tiny,           % размер шрифта для номеров строк
		stepnumber=1,                   % размер шага между двумя номерами строк
		numbersep=5pt,                % как далеко отстоят номера строк от подсвечиваемого кода
		backgroundcolor=\color{white}, % цвет фона подсветки - используем \usepackage{color}
		showspaces=false,            % показывать или нет пробелы специальными отступами
		showstringspaces=false,      % показывать или нет пробелы в строках
		showtabs=false,             % показывать или нет табуляцию в строках
		frame=single,              % рисовать рамку вокруг кода
		tabsize=2,                 % размер табуляции по умолчанию равен 2 пробелам
		captionpos=t,              % позиция заголовка вверху [t] или внизу [b] 
		breaklines=true,           % автоматически переносить строки (да\нет)
		breakatwhitespace=false, % переносить строки только если есть пробел
		escapeinside={\%*}{*)},  % если нужно добавить комментарии в коде
		caption=Функция для вычисление катета по заданной гипотенузе и другому катету в прямоугольном треугольнике.
		]{./first-middle3}
\end{itemize}