\documentclass[a4paper,14pt]{report} %размер бумаги устанавливаем А4, шрифт 14 пунктов
\usepackage[T2A]{fontenc}
\usepackage[english,russian]{babel} % включаем русский язык
\usepackage[utf8x]{inputenc} % работаем с форматом  utf-8
\usepackage{tocloft}% изменяем параметры оглавления
\usepackage{graphicx} % библиотека для работы с графикой
\graphicspath{ {./img/} } % путь к папке с изображениями
\usepackage{amssymb,amsfonts,amsmath,mathtext,cite,enumerate,float} %подключаем нужные пакеты расширений
\usepackage{cite} % Красивые ссылки на литературу
\usepackage{mathtext}
\usepackage{caption} % описания
\usepackage{color} %% это для отображения цвета в коде
\usepackage{listings} %% собственно, это и есть пакет listings
\linespread{1.3}
\usepackage{geometry} % Меняем поля страницы
\geometry{left=2cm}% левое поле
\geometry{right=1.5cm}% правое поле
\geometry{top=1cm}% верхнее поле
\geometry{bottom=2cm}% нижнее поле
%%% Стили %%%
\bibliographystyle{BibTeX-Styles/utf8gost71u}    % Оформляем библиографию по ГОСТ 7.1 (ГОСТ Р 7.0.11-2011, 5.6.7)
\cftsetpnumwidth{3pt} % точки прям до цифр в оглавлении

\makeatletter
\renewcommand{\@biblabel}[1]{#1.} % Заменяем библиографию с квадратных скобок на точку:
\makeatother

\renewcommand{\theenumi}{\arabic{enumi}}% Меняем везде перечисления на цифра.цифра
\renewcommand{\labelenumi}{\arabic{enumi})}% Меняем везде перечисления на цифра.цифра
\renewcommand{\theenumii}{.\arabic{enumii}}% Меняем везде перечисления на цифра.цифра
\renewcommand{\labelenumii}{\arabic{enumi}.\arabic{enumii}.}% Меняем везде перечисления на цифра.цифра
\renewcommand{\theenumiii}{.\arabic{enumiii}}% Меняем везде перечисления на цифра.цифра
\renewcommand{\labelenumiii}{\arabic{enumi}.\arabic{enumii}.\arabic{enumiii}.}% Меняем везде перечисления на цифра.цифра
\renewcommand{\thesection}{\arabic{section}.}
\renewcommand{\thesubsection}{\arabic{section}.\arabic{subsection}.}
\newcommand{\tocsecindent}{\hspace{7mm}}
\newcommand*{\No}{\textnumero}
\definecolor{gray}{rgb}{0.5,0.5,0.5}
\DeclareCaptionFont{white}{\color{white}} %% это сделает текст заголовка белым
%% код ниже нарисует серую рамочку вокруг заголовка кода.
\DeclareCaptionFormat{listing}{\colorbox{gray}{\parbox{\textwidth}{#1#2#3}}}
\captionsetup[lstlisting]{format=listing,labelfont=white,textfont=white}
\addto\captionsrussian{\def\refname{Список литературы}}
\long\def\comment{}% многострочный комментарий

% документ
\begin{document}
\long\def\/*#1*/{}
	\input{title} % титульный лист
	\newpage
	\tableofcontents % это оглавление, которое генерируется автоматически
	% задание и сама лабораторная работа
\newpage
\section*{Задание}
\addcontentsline{toc}{section}{\tocsecindent{Задание}}
\Large{Лабораторная работа №1}\\

\large{Представить следующие списки в виде списочных ячеек:}\\
\\
1) '(open close halph) \hspace{5cm}4) '((TOOL) (call))\\
2) '((open1) (close2) (halph3)) \hspace{3.4cm}5) '((TOOL1) ((call2)) ((sell)))\\
3) '((one) for all (and(me(for you)))) \hspace{2.3cm}6) '(((TOOL) (call)) ((sell)))\\


\begin{minipage}[t]{1.0\textwidth}
  \centering\raisebox{\dimexpr \topskip-\height}{%
  \includegraphics[width=\textwidth]{1.png}}
  \captionof{figure}{'(open close halph)}
  \label{fig1}
\end{minipage}\hfill

\begin{minipage}[t]{1.0\textwidth}
  \centering\raisebox{\dimexpr \topskip-\height}{%
  \includegraphics[width=\textwidth]{2.png}}
  \captionof{figure}{'((open1) (close2) (halph3)) }
  \label{fig2}
\end{minipage}\hfill

\begin{minipage}[t]{1.0\textwidth}
  \centering\raisebox{\dimexpr \topskip-\height}{%
  \includegraphics[width=\textwidth]{3.png}}
  \captionof{figure}{'((one) for all (and(me(for you))))}
  \label{fig3}
\end{minipage}\hfill

\begin{minipage}[t]{1.0\textwidth}
  \centering\raisebox{\dimexpr \topskip-\height}{%
  \includegraphics[width=\textwidth]{4.png}}
  \captionof{figure}{'((TOOL) (call))}
  \label{fig4}
\end{minipage}\hfill

\begin{minipage}[t]{1.0\textwidth}
  \centering\raisebox{\dimexpr \topskip-\height}{%
  \includegraphics[width=\textwidth]{5.png}}
  \captionof{figure}{'((TOOL1) ((call2)) ((sell)))}
  \label{fig5}
\end{minipage}\hfill

\begin{minipage}[t]{1.0\textwidth}
  \centering\raisebox{\dimexpr \topskip-\height}{%
  \includegraphics[width=\textwidth]{6.png}}
  \captionof{figure}{'(((TOOL) (call)) ((sell)))}
  \label{fig6}
\end{minipage}\hfill


	% ответы на вопросы
\newpage
\section*{Ответы на вопросы}
\addcontentsline{toc}{section}{\tocsecindent{Ответы на вопросы}}
\begin{enumerate}
  \item\Large{Как воспринимается символ '? Дать определение списка.}
    \begin{itemize}
    Функция quote блокирует вычисления(предохраняет свой единственный аргумент от вычисления):
    \begin{itemize}
    	\item (quote (a b c)) блокирует вычисление a b c\\
	(a b c)  система должна вычислить a b c
	\item '(a b c) аналог quote(блокирование запуска eval)\\
	a b c тогда воспринимаются как данные, а не как фрагмент программы
    \end{itemize}
      \item \underline{Список}- рекурсивно определенная динамическая структура, которая может быть пустой.
      \item \underline{Список}- динамическая структура, которая может быть пустой, имеет голову и хвост
    \end{itemize}
  \item\Large{Как представляются списки в ОП? Выписать и дать определение основным элементам языка Lisp.}\\
  Списочная ячейка состоит из двух частей, полей first и rest. Каждое из полей содержит указатель. Указатель может ссылаться на другую списочную ячейку или на некоторый другой Lisp объект, как, например, атом. Указатели между ячейками образуют как бы цепочку, по которой можно из предыдущей ячейки попасть в следующую и так, наконец, до атомарных объектов. Каждый известный системе атом записан в определённом месте памяти лишь один раз. (
  \\
    \underline{Основные элементы языка:}
    \begin{itemize}
      \item Атомы:
      \begin{itemize}
          \item Символ
          \item Символьная константа
	  \item Число(целые, действительные, рациональные)
	  \item Логические(T, Nil)
      \end{itemize}
      \item Точечные пары:
      \begin{itemize}
      	   \item (<атом> . <атом>)
	   \item (<точ. пара> . <атом>)
	   \item (<атом> . <точ. пара>)
	   \item (<точ. пара> . <точ. пара>)
      \end{itemize}
      \item S-выражения:
      \begin{itemize}
      	   \item <атом> | <точ.пара>
      \end{itemize}
      \item Списки:
      \begin{itemize}
      	   \item (<s-выражение> . <список>)
	   \item (<пустой список>) $\equiv$ Nil
      \end{itemize}
    \end{itemize}
\end{enumerate}

% ответ
%\Large{3) Как выполняются CAR и CDR и какой результат они вернут в разных случаях?}
%\\
	% список литературы
\newpage
\addcontentsline{toc}{section}{\tocsecindent{Список литературы}}

\begin{thebibliography}{}
    \bibitem{litlink1}  Толпинская Н.Б.  -  Курс лекций по "Функциональному и Логическому программированию"[Текст], Москва 2019 год.
    \bibitem{litlink2}  Городняя Л.В.  -  "Основы функционального программирования. Курс лекций" [Текст] / Л.В. Городняя.    – Москва: Интернет-университет информационных технологий,  2004.
\end{thebibliography}

\end{document}