% задание и сама лабораторная работа
\newpage
\section*{Задание}
\addcontentsline{toc}{section}{\tocsecindent{Задание}}
\Large{Лабораторная работа №4}

\begin{itemize}
	\item Составить диаграмму вычисления следующих выражений:
		\begin{enumerate}
			\item (equal 3 abs(-3)) = T
			\item (equal (+ 1 2) 3) = T
			\item (equal (* 4 7) 21) = Nil
			\item (equal (* 2 3) (+ 7 2)) = Nil
			\item (equal (- 7 3) (* 3 2)) = Nil
			\item (equal abs((- 2 4)) 3) = Nil
		\end{enumerate}
	\item Написать функцию, которая вычисляет катет по заданной гипотенузе и другому катету прямоугольного треугольника, и составить диаграмму ее вычисления.
		\lstinputlisting[
		language=Lisp,                 % выбор языка для подсветки (здесь это С)
		basicstyle=\small\sffamily, % размер и начертание шрифта для подсветки кода
		numbers=left,               % где поставить нумерацию строк (слева\справа)
		numberstyle=\tiny,           % размер шрифта для номеров строк
		stepnumber=1,                   % размер шага между двумя номерами строк
		numbersep=5pt,                % как далеко отстоят номера строк от подсвечиваемого кода
		backgroundcolor=\color{white}, % цвет фона подсветки - используем \usepackage{color}
		showspaces=false,            % показывать или нет пробелы специальными отступами
		showstringspaces=false,      % показывать или нет пробелы в строках
		showtabs=false,             % показывать или нет табуляцию в строках
		frame=single,              % рисовать рамку вокруг кода
		tabsize=2,                 % размер табуляции по умолчанию равен 2 пробелам
		captionpos=t,              % позиция заголовка вверху [t] или внизу [b] 
		breaklines=true,           % автоматически переносить строки (да\нет)
		breakatwhitespace=false, % переносить строки только если есть пробел
		escapeinside={\%*}{*)},  % если нужно добавить комментарии в коде
		caption=Функция для вычисление катета по заданной гипотенузе и другому катету в прямоугольном треугольнике.
		]{./leg}
	\item Написать функцию, которая вычисляет площадь трапеции по ее основаниям и высоте, и составить диаграмму ее вычисления.
		\lstinputlisting[
		language=Lisp,                 % выбор языка для подсветки (здесь это С)
		basicstyle=\small\sffamily, % размер и начертание шрифта для подсветки кода
		numbers=left,               % где поставить нумерацию строк (слева\справа)
		numberstyle=\tiny,           % размер шрифта для номеров строк
		stepnumber=1,                   % размер шага между двумя номерами строк
		numbersep=5pt,                % как далеко отстоят номера строк от подсвечиваемого кода
		backgroundcolor=\color{white}, % цвет фона подсветки - используем \usepackage{color}
		showspaces=false,            % показывать или нет пробелы специальными отступами
		showstringspaces=false,      % показывать или нет пробелы в строках
		showtabs=false,             % показывать или нет табуляцию в строках
		frame=single,              % рисовать рамку вокруг кода
		tabsize=2,                 % размер табуляции по умолчанию равен 2 пробелам
		captionpos=t,              % позиция заголовка вверху [t] или внизу [b] 
		breaklines=true,           % автоматически переносить строки (да\нет)
		breakatwhitespace=false, % переносить строки только если есть пробел
		escapeinside={\%*}{*)},  % если нужно добавить комментарии в коде
		caption=Функция для вычисление площади трапеции по ее основниям и высоте.
		]{./square}
\end{itemize}